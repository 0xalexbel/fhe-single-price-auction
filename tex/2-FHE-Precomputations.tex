\section{FHE precomputations}

% =============================================================================
% Bid validation
% =============================================================================

\subsection{Bid validation}

\subsubsection{Upper bounds}
To ensure that no operation results in arithmetic overflow, the following conditions must be satisfied:
\begin{equation*}
    \begin{split}
        N &< 2^{16} \\
        \sum_{i=1}^{N} q_i &< 2^{256}
    \end{split}
\end{equation*}
which can be simplified into the stricter condition:
\begin{equation*}
    N < 2^{16} \ \text{ and } \ Q < 2^{240} \ \text{ and } \ \forall i \in I_N, \ q_i \le Q  
\end{equation*}
By imposing this condition on each bidder, we can handle the worst-case scenario without arithmetic overflow:
\begin{equation*}
    \begin{cases}
        N = 2^{16}-1 &\text { unique bidders participating in the auction, } \\
        Q = 2^{240}-1 &\text { tokens available for sale in the auction, } \\
        q = 2^{240}-1 &\text { quantity of tokens bid by each bidder. } \\
    \end{cases}
\end{equation*}

\subsubsection{Quantity clamping}
Ensure that bid quantities do not exceed the total available quantity $Q$:
\begin{equation*}
    \forall i \in I_N, \quad q_i := \min(Q, q_i) 
\end{equation*}

\subsubsection{Validation}
Update bids to reflect validity conditions:
\begin{equation*}
    (p_i, q_i) := 
    \begin{cases}
        (0, 0), \quad &\text{if } p_i = 0 \text{ or } q_i = 0, \\
        (p_i, q_i), \quad &\text{otherwise.}
    \end{cases}
\end{equation*}

% =============================================================================
% Price Matrices
% =============================================================================

\subsection{Price Matrices $\mathbf{P_{eq}}$, $\mathbf{P_{ge}}$, $\mathbf{P_{gt}}$}

\subsubsection{Definitions}
\begin{itemize}
    \item Let $\mathbf{P_{eq}} = (\mathbf{P_{eq}}[i, \ j])_{\ 1 \le i, j \le N}$ denote the price equality matrix on $\mathcal{B}$, whose entries are defined as follows:
    \begin{equation*}
        \forall i,j \in I_N, \quad \mathbf{P_{eq}}[i, j] = 
        \begin{cases}
            1  \quad \text{if } \ p_{i} = p_{j} \\
            0  \quad \text{otherwise} \\
        \end{cases}
    \end{equation*}        
    \item Let $\mathbf{P_{ge}} = (\mathbf{P_{ge}}[i, \ j])_{\ 1 \le i, j \le N}$ denote the price comparison matrix on $\mathcal{B}$, whose entries are defined as follows:
    \begin{equation*}
        \forall i,j \in I_N, \quad \mathbf{P_{ge}}[i, j] = 
        \begin{cases}
            1  \quad \text{if } \ p_{i} \ge p_{j} \\
            0  \quad \text{otherwise} \\
        \end{cases}
    \end{equation*}
    \item Let $\mathbf{P_{gt}} = (\mathbf{P_{gt}}[i, \ j])_{\ 1 \le i, j \le N}$ denote the price comparison matrix on $\mathcal{B}$, whose entries are defined as follows:
    \begin{equation*}
        \forall i,j \in I_N, \quad \mathbf{P_{gt}}[i, j] = 
        \begin{cases}
            1  \quad \text{if } \ p_{i} > p_{j} \\
            0  \quad \text{otherwise} \\
        \end{cases}
    \end{equation*}
\end{itemize}
Which can be simplified:    
\begin{equation*}
\begin{split}
    \mathbf{P_{eq}}[i, j] &= 
    \begin{cases}
        1              \quad &\text{if } \ i = j \\
        p_{i} = p_{j}  \quad &\text{if } \ i < j \\
        \mathbf{P_{eq}}[j, i] \quad &\text{if } \ i > j \\
    \end{cases}
    \\
    \\
    \mathbf{P_{gt}}[i, j] &= 
    \begin{cases}
        0              \quad &\text{if } \ i = j \\
        p_{i} > p_{j}  \quad &\text{if } \ i < j \\
        \neg(\mathbf{P_{gt}}[j, i] \lor \mathbf{P_{eq}}[i, j])  \quad &\text{if }  \ i > j
    \end{cases}
    \\
    \\
    \mathbf{P_{ge}}[i, j] &= 
    \begin{cases}
        1              \quad &\text{if } \ i = j \\
        \mathbf{P_{gt}}[j, i] \lor \mathbf{P_{eq}}[j, i] \quad &\text{if }  \ i > j
    \end{cases}
\end{split}
\end{equation*}

\subsubsection{FHE Cost}

\renewcommand{\arraystretch}{1.5}
\begin{tabular}{ |l|c|c|c| }
    \hline    
    Operations & $\mathbf{P_{eq}}$ & $\mathbf{P_{ge}}$ & $\mathbf{P_{gt}}$ \\ 
    \hline
    fheEq(U256)         & $N(N-1)/2$  & $0$      & $0$ \\
    fheGt(U256)         & $0$         & $0$      & $N(N-1)/2$ \\
    fheOr(Bool)         & $0$         & $N(N-1)$ & $N(N-1)/2$ \\
    fheNot(Bool)        & $0$         & $0$      & $N(N-1)/2$ \\
    \hline
    \hline
    FHE Units           & $2N(N-1)$   & $N(N-1)$ & $11N(N-1)/2$ \\
    \hline
\end{tabular}

% =============================================================================
% Quantity Matrices
% =============================================================================

\subsection{Quantity Matrices $\mathbf{Q_{eq}}$, $\mathbf{Q_{ge}}$, $\mathbf{Q_{gt}}$}
Similarly, we define the quantity matrices $\mathbf{Q_{eq}}$, $\mathbf{Q_{gt}}$ and $\mathbf{Q_{ge}}$ in the same manner.

% =============================================================================
% Random Matrix
% =============================================================================

\subsection{Random Matrix $\mathbf{Rand_{gt}}$}

To perform the auction allocation using price and randomization, we assign each bidder $B_i$ a unique random value to serve as a tie-breaking rule between bidders.

\subsubsection{Definition}

\begin{itemize}
\item Let $rand(i)$ denote a bijective function that assigns a unique random value to each bidder $B_i$. It is formally defined as follows:
\begin{equation*}
    rand: I_N \to I_N \\
    \forall i,j \in I_N, \quad i = j \iff rand(i) = rand(j)
\end{equation*}

\item Let $\mathbf{Rand_{gt}} = (\mathbf{Rand_{gt}}[i, \ j])_{\ 1 \le i, j \le N}$ denote random value comparison matrix on $\mathcal{B}$. The entries $\mathbf{Rand_{gt}}[i, j]$ are defined as follows:
\begin{equation*}
    \forall i,j \in I_N, \quad \mathbf{Rand_{gt}}[i, j] = 
    \begin{cases}
        1  \quad \text{if } \ rand(i) > rand(j) \\
        0  \quad \text{otherwise} \\
    \end{cases}
\end{equation*}
\end{itemize}

\subsubsection{FHE Fisher-Yates Shuffle Algorithm}
One way to compute the matrix $\mathbf{Rand_{gt}}$ is by shuffling the set $\{ 1, 2, \dots, N \}$ using the $O(N)$ Fisher-Yates shuffle algorithm.

\subsubsection{Solidity sample code}

\begin{lstlisting}[language=Solidity]

    // returns true if i is a power of 2
    function isPowerOfTwo(uint16 i) returns (bool);
    
    function swap(uint16 a, euint16 b, euint16[] memory arr) {
        for(uint16 i = 0; i < N; ++i) {
            ebool b_eq_i = TFHE.eq(b, i);
            // arr[b] = arr[a];
            arr[i] = TFHE.ifThenElse(b_eq_i, arr[a], arr[i]);
            // arr[a] = arr[b]
            arr[a] = TFHE.ifThenElse(b_eq_i, arr[i], arr[a]);
        }
    }
    
    function shuffle(euint16[] memory arr) {
        for(uint16 i = N-1; i >= 1; --i) {
            euint16 j;
            // j = random integer such that 0 <= j <= i
            if (isPowerOfTwo(i+1)) {
                // returns [0, i+1) = [0, i]
                j = TFHE.randEuint16(i+1);
            } else {
                euint16 rnd = TFHE.randEuint16();
                // returns [0, i+1) = [0, i]
                j = TFHE.rem(rnd, i+1);
            }
            swap(i, j, arr);
        }
    }
    
    function rand(uint16 i) returns(euint16) {
        return shuffledArray[i];
    }
        
\end{lstlisting}

\subsubsection{FHE Cost}

\renewcommand{\arraystretch}{1.5}
\begin{tabular}{ |l|c|c|c| }
    \hline    
    Operations & swap & shuffle & $\mathbf{Rand_{gt}}$ \\ 
    \hline
    fheEq(U16)          & $N$   & $0$      & $0$ \\
    fheIfThenElse(U16)  & $2N$  & $0$      & $0$ \\
    fheRand(U16)        & $0$   & $N-1$    & $0$ \\
    fheRem(U16)         & $0$   & $N-1$    & $0$ \\
    fheGt(U16)          & $0$   & $0$      & $N^2$ \\
    \hline
    \hline
    FHE Units           & $6N$  & $(6N + 28)(N-1)$ & $2N^2$ \\
    \hline
\end{tabular}

% =============================================================================
% Bid Order Comparison Matrix
% =============================================================================

\subsection{Bid Order Comparison Matrix $\mathbf{B_{gt}}$}

Given a strict order relation $>$ on $\mathcal{B}$, we define $\mathbf{B_{gt}} = (\mathbf{B_{gt}}[i, j])_{1 \le i, j \le N}$ as the \textbf{binary comparison matrix} associated with $>$. The entries $\mathbf{B_{gt}}[i, j]$ are defined as follows:

\begin{equation*}
    \forall i, j \in I_N, \quad \mathbf{B_{gt}}[i, j] = 
    \begin{cases}
        1  & \text{if } B_i > B_j, \\
        0  & \text{otherwise}.
    \end{cases}
\end{equation*}

\subsubsection{Order by Price, Quantity, and Bid Placement}

\begin{equation*}
    \forall i,j \in I_N, \quad \mathbf{B_{gt}}[i, j] = 
    \begin{cases}
        0  \quad &\text{if } \ i = j \\
        \mathbf{P_{ge}}[i, j] \ \land \ [ \ \mathbf{P_{gt}}[i, j] \ \lor \ \mathbf{Q_{ge}}[i, j] \ ]  \quad &\text{if } \ i < j \\
        \neg \mathbf{B_{gt}}[j, i] \quad &\text{if } \ i > j
    \end{cases}
\end{equation*}

\setlength{\parindent}{0pt}
\renewcommand{\arraystretch}{1.5}
\begin{tabular}{ |l|c|c|c| }
    \hline    
    Operations & $\mathbf{B_{gt}}$ \\ 
    \hline
    fheOr(bool)          & $N(N-1)/2$   \\
    fheAnd(bool)         & $N(N-1)/2$   \\
    fheNot(bool)         & $N(N-1)/2$   \\
    \hline
    \hline
    FHE Units           & $3N(N-1)/2$  \\
    \hline
\end{tabular}

\subsubsection{Order by Price and Bid Placement}

\begin{equation*}
    \forall i,j \in I_N, \quad \mathbf{B_{gt}}[i, j] = 
    \begin{cases}
        0  \quad &\text{if } \ i = j \\
        \mathbf{P_{ge}}[i, j] \quad &\text{if } \ i < j \\
        \mathbf{P_{gt}}[i, j] \quad &\text{if } \ i > j
    \end{cases}
\end{equation*}

The additionnal FHE-cost is null when bidders are sorted using price and bid placement.

\subsubsection{Order by Price and Randomization}

Using the $\mathbf{Rand_{gt}}$ matrix defined above, the comparison matrix can be computed as follows:

\begin{equation*}
    \forall i,j \in I_N, \quad \mathbf{B_{gt}}[i, j] = 
    \begin{cases}
        0  \quad &\text{if } \ i = j \\
        \mathbf{P_{ge}}[i, j] \land [ \ \mathbf{P_{gt}}[i, j] \lor \mathbf{Rand_{gt}}[i, j] \ ] \quad &\text{if } \ i < j \\
        \neg \mathbf{B_{gt}}[j, i] \quad &\text{if } \ i > j
    \end{cases}
\end{equation*}

\setlength{\parindent}{0pt}
\renewcommand{\arraystretch}{1.5}
\begin{tabular}{ |l|c|c|c| }
    \hline    
    Operations & $\mathbf{B_{gt}}$ \\ 
    \hline
    fheOr(bool)          & $N(N-1)/2$   \\
    fheAnd(bool)         & $N(N-1)/2$   \\
    fheNot(bool)         & $N(N-1)/2$   \\
    \hline
    \hline
    FHE Units           & $3N(N-1)/2$  \\
    \hline
\end{tabular}
