\section{FHE sort}

% =============================================================================
% Rank function
% =============================================================================

\subsection{Rank function}

\subsubsection{Definition}
Let $rank: \{1, 2, \dots, N\} \to \{0, 1, \dots, N^{*}-1\}$ denote the \textbf{rank function} under the strict order relation $>$ defined on $\mathcal{B}$, which assigns a rank value to each bidder index $i$. Specifically, $rank$ represents the position of $B_i$ in the descending order of the set $\mathcal{B}$. Formally, the rank of bidder $B_i$ is given by:

\begin{equation*}
    rank(i) = \left| \{ B_j \in \mathcal{B} \mid B_j > B_i \} \right|
\end{equation*}

Where:
\begin{itemize}
    \setlength\itemsep{0em}
    \item[--]$\left| \cdot \right|$ denotes the cardinality of the set.
    \item[--]$rank(i) = 0$ if $B_i$ is the largest element under $>$,
    \item[--]$rank(i) = N^{*}-1$ if $B_i$ is the smallest element under $>$.
    \item[--]If $N^{*} = N$ then each bidder has a unique rank and the auction has no tie under $>$.
    \item[--]If $N^{*} < N$ then two or more distinct bidders are sharing the same rank, the auction has one or more ties.
\end{itemize}

Thus, the $rank$ function can also be written as follows using the comparison matrix $\mathbf{B_{gt}}$:

\begin{equation*}
    rank(i) = \sum_{\substack{j=1 \\ j \neq i }}^{N} \mathbf{B_{gt}}[j, i]
\end{equation*}

\subsubsection{Unique rankings}
When $>$ produces a set of bidders with unique rankings (i.e. no ties), the rank function $\text{rank}(i)$ becomes bijective. 

% =============================================================================
% Rank vector
% =============================================================================

\subsection{Rank vector $\mathbf{rank}$}

\subsubsection{Definition}

Let $\mathbf{rank} = (\mathbf{rank}[i])_{1 \le i \le N}$ denote the vector whose entries $\mathbf{rank}[i]$ are defined as:

\begin{equation*}
    \forall i, k \in \{1, 2, 3, \dots, N\}, \quad \mathbf{rank}[i] = rank(i)
\end{equation*}

\subsubsection{FHE cost}

\renewcommand{\arraystretch}{1.5}
\begin{tabular}{ |l|c|c| }
    \hline    
    Operations & $rank(i)$ & $\mathbf{rank}$ \\ 
    \hline
    fheAdd(U16)          & $N-1$ & $N(N-1)$  \\
    \hline
    \hline
    FHE Units           & $5(N-1)$ & $5N(N-1)$ \\
    \hline
\end{tabular}

\subsubsection{Solidity sample code}

\begin{lstlisting}[language=Solidity]

  function rankAt(uint16 i) returns(euint16 rank) {
      rank = TFHE.asEuint16(0);
      for(uint16 j = 0; j < N; ++j) {
          if (i != j) {
              rank = TFHE.add(r, TFHE.asEuint16(Bgt(j,i)));
          }
      }
  }

\end{lstlisting}

% =============================================================================
% Rank Matrix
% =============================================================================

\subsection{Rank Matrix $\mathbf{R_{eq}}$}

\subsubsection{Definition}
Let $\mathbf{R_{eq}} = (\mathbf{R_{eq}}[i, k])_{1 \le i, k \le N}$ denote the matrix whose entries $\mathbf{R_{eq}}[i, k]$ are defined as:
\begin{equation*}
    \forall i, k \in \{1, 2, 3, \dots, N\}, \quad \mathbf{R_{eq}}[i, k] = 
    \begin{cases}
        1, & \text{if } \ \mathbf{rank}[i] = k - 1, \\
        0, & \text{otherwise}.
    \end{cases}
\end{equation*}

\subsubsection{FHE cost}

\renewcommand{\arraystretch}{1.5}
\begin{tabular}{ |l|c| }
    \hline    
    Operations & $\mathbf{R_{eq}}$ \\ 
    \hline
    fheEq(U16) & $N^2$  \\
    \hline
    \hline
    FHE Units & $2N^2$ \\
    \hline
\end{tabular}

\subsubsection{Solidity sample code}

\begin{lstlisting}[language=Solidity]

    function Req(uint16 i, uint16 k) returns(ebool eq) {
        eq = TFHE.eq(rankAt(i), k);
    }
  
\end{lstlisting}
  
% =============================================================================
% Quantity Vector
% =============================================================================

\subsection{Quantity vector $\mathbf{q_{rank}}$}

\subsubsection{Definition}
Let $\mathbf{q_{rank}} = (\mathbf{q_{rank}}[k])_{1 \le k \le N}$ denote the vector where the $k$-th entry represents the total quantity of tokens bid by all the bidders ranked at the $k$-th position upon completion of the auction.
The entries of $\mathbf{q_{rank}}$ are defined as follows:
\begin{equation*}
\forall k \in \{1, 2, \dots, N\}, \quad \mathbf{q_{rank}}[k] = \sum_{\substack{i \in \{1, 2, \dots, N\}}} \mathbf{R_{eq}}[i, k] . q_i
\end{equation*}

\subsubsection{Case with unique rankings}
When $>$ produces a set of bidders with unique rankings (i.e. no ties), the rank function $\text{rank}(i)$ becomes bijective. In this case the vector $\mathbf{q_{rank}}$ can be expressed using bitwise operations resulting in a more efficient formula in terms of FHE cost:
\begin{equation*}
\forall k \in \{1, 2, \dots, N\}, \quad \mathbf{q_{rank}}[k] = \bigvee_{\substack{i \in \{1, 2, \dots, N\}}} \mathbf{R_{eq}}[i, k] \land q_i
\end{equation*}

\subsubsection{FHE cost}

\renewcommand{\arraystretch}{1.5}
\begin{tabular}{ |l|c|c|c|c| }
    \hline    
    Operations & $\mathbf{q_{rank}}[k] (unique)$ & $\mathbf{q_{rank}} (unique)$ & $\mathbf{q_{rank}}[k] (ties)$ & $\mathbf{q_{rank}} (ties)$ \\ 
    \hline
    fheAnd(U256) & $N$ & $N^2$ & $0$ & $0$ \\
    fheOr(U256) & $N-1$ & $N(N-1)$ & $0$ & $0$ \\
    fheAdd(U256) & $0$ & $0$ & $N-1$ & $N(N-1)$ \\
    fheIfThenElse(U256) & $0$ & $0$ & $N$ & $N^2$ \\
    \hline
    \hline
    FHE Units & $4N-2$ & $N(4N-2)$ & $14N-10$ & $N(14N-10)$ \\
    \hline
\end{tabular}

\subsubsection{Solidity sample code}

\begin{lstlisting}[language=Solidity]

    function Req(uint16 bi, uint16 k) returns(ebool);
    function quantity(uint16 bi) returns (euint256);

    // If bidders can have identical rankings (i.e., ties exist)
    function qRankWithTies(uint16 k) returns(euint256 qRank) {
        qRank = TFHE.ifThenElse(Req(0,k), quantity(0), TFHE.asEuint256(0));
        for(uint16 i = 1; i < N; ++i) {
            euint256 q = TFHE.ifThenElse(Req(i,k), quantity(i), TFHE.asEuint256(0));
            qRank = TFHE.add(qRank, q);
        }
    }
\end{lstlisting}

\begin{lstlisting}[language=Solidity]

    // returns 0 or 2^256-1
    function Req(uint16 bi, uint16 k) returns(euint256);
    function quantity(uint16 bi) returns (euint256);

    // If bidders have unique rankings (i.e., there are no ties)
    function qRankUnique(uint16 k) returns(euint256 qRank) {
        qRank = TFHE.and(Req(0,k), quantity(0));
        for(uint16 i = 1; i < N; ++i) {
            euint256 q = TFHE.and(Req(i,k), q(i));
            qRank = TFHE.or(qRank, q);
        }
    }    
\end{lstlisting}

% =============================================================================
% Price Vector
% =============================================================================

\subsection{Price vector $\mathbf{p_{rank}}$}

\subsubsection{Definition}

Let $\mathbf{p_{rank}} = (\mathbf{p_{rank}}[k])_{1 \le k \le N}$ denote the vector whose $k$-th entry represents the common token unit price bid by each of $k$-th highest-ranked bidders upon completion of the auction.

The entries of $\mathbf{p_{rank}}$ can be computed as follows:

\begin{equation*}
    \forall k \in \{1, 2, \dots, N\}, \quad \mathbf{p_{rank}}[k] = \bigvee_{\substack{i \in \{1, 2, \dots, N\}}} \mathbf{R_{eq}}[i, k] \land p_i
\end{equation*}

\subsubsection{FHE cost}

\renewcommand{\arraystretch}{1.5}
\begin{tabular}{ |l|c|c|c|c| }
    \hline    
    Operations & $\mathbf{p_{rank}}[k] (bitwise)$ & $\mathbf{p_{rank}} (bitwise)$ & $\mathbf{p_{rank}}[k] (if/then/else)$ & $\mathbf{p_{rank}} (if/then/else)$ \\ 
    \hline
    fheAnd(U256) & $N$ & $N^2$ & $0$ & $0$ \\
    fheOr(U256) & $N-1$ & $N(N-1)$ & $0$ & $0$ \\
    fheIfThenElse(U256) & $0$ & $0$ & $N$ & $N^2$ \\
    \hline
    \hline
    FHE Units & $4N-2$ & $N(4N-2)$ & $4N$ & $4N^2$ \\
    \hline
\end{tabular}

\subsubsection{Solidity sample code}

\begin{lstlisting}[language=Solidity]

    function Req(uint16 bi, uint16 k) returns(ebool);
    function price(uint16 bi) returns (euint256);

    function pRank(uint16 k) returns(euint256) {
        euint256 p = TFHE.ifThenElse(Req(0,k), price(0), TFHE.asEuint256(0));
        for(uint16 i = 1; i < N; ++i) {
            p = TFHE.ifThenElse(Req(i,k), price(i), p);
        }
        return p;
    }
\end{lstlisting}

\begin{lstlisting}[language=Solidity]

    // returns 0 or 2^256-1
    function Req(uint16 bi, uint16 k) returns(euint256);
    function price(uint16 bi) returns (euint256);

    function pRank(uint16 k) returns(euint256 p) {
        euint256 p = TFHE.and(Req(0,k), price(0));
        for(uint16 i = 1; i < N; ++i) {
            euint256 _p = TFHE.and(Req(i,k), q(i));
            p = TFHE.or(p, _p);
        }
        return p;
    }    

\end{lstlisting}
